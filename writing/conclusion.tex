% !TeX root = main.tex
\section{总结}
在工程经济学的学习过程中, 我们在回归分析法的自学过程中产生了从区位视角, 运用模型分析房价的这样一种想法.

从眼就开始到结束的过程中, 我们的模型也发生了很大的变化.
一开始, 我们的研究方法是考虑一系列小区到某几个重要的区位点的距离 --- 如万象汇, 故宫博物院, 在拟合的过程中, 我们发现数据结果并不理想, 于是我们从前面提到的两篇硕士论文中寻找思路, 运用高德开放平台爬取大量 POI 点, 再利用模型进行分析.
一开始, 我们的模型算法也仅仅停留在最近点或是同类中心点距离的和, 由于 $R^2$ 较低, 我们不断调整算法, 最终使得 $R^2$ 达到比较好的水平, 同时这种算法对应的现实意义也是优于前两种的.

经过分析, 我们得到了价格热力图, 各个区位的影响因子, 并用这样的模型来估测 2018 与 2022 年的住房价格, 得到了比较好的结果.

其实, 这样的研究也是具有一定现实意义的.
正如老师在课堂上讲授的一样, 许多公共投资项目需要考虑其产生的社会总效益, 而房价正是反映人们对于某一建筑的支付意愿的一个窗口;同时, 这样的研究也可以为我们辅助我们预测房价.

土木学科叫做 Civil Engineering, 也就是民用工程.
这是一个与人的生活息息相关的学科, 实际地服务于社会, 服务于人们的日常生活, 是这个学科的终极意义, 这也是我们在工程经济学课上所体悟到的价值内核.
