% !TeX root = main.tex
\section{相关工作}
在研究某一地区住房价格影响因素的过程中, 许多研究者采用回归分析法, GWR 模型法等方法进行研究.
由于本文的模型是考虑区位对房价的影响情况, 本部分主要关注那些采用区位的视角来考察房价的研究.
西南大学的刘青霞, 王方民两人在其硕士论文中都利用了回归分析法对某个特定的因素对于房价的影响做了估计. \cite{RN4,RN5}

阅读文章, 可以发现刘青霞建立的模型在一些方面合理且细致, 但不难发现, 研究的结果并不如人意.
例如, 我们发现刘青霞的回归分析模型 $R^2$ 只有 \num{0.5415}, 可以发现, 这一方面显然来源于其模型的过度简化.

在刘的研究中, 对于所有的影响因素, 都统一取了该居民小区到距离其最近的该类建筑的距离, 而这种过度的简化其实是不合理的.
首先, ``取最小值'' 的方法本身对于某些影响因素是不合理的.
例如, 在研究中, 作者对于居民建筑与所有火车站之间的距离取最小值.
然而, 无论是在大家购买房子的时候, 还是在实际使用的过程中, 居民实际前往哪一个火车站是基于特定列车的始发车站的, 与``最近的火车站距离'' 其实没有直接关系.
其次, ``取最小值'' 的处理就导致在其研究中, 所有的模型在组内都是不分排名, 不加权重的.
比如, 在模型中所有的轨道交通站都是一致的.
然而, 在现实生活中, 多条线路交汇的轨道交通站与只有一条线路的轨道交通站对于周围房价的影响一定是有不同的.

对于以上两个方面, 如果模型如果能改成对于一定区域内的轨交车站, 火车站按照其重要性取加权平均, 那么就会显得更加合理.
